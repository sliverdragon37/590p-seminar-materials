\documentclass{beamer}
\usetheme{default}

\newcommand{\defeq}{\overset{\text{def}}\equiv}

\begin{document}
\begin{centering}
\begin{frame}
Reasoning About Multi-Stage Programs\\
Jun Inoue and Walid Taha\\
\end{frame}

\begin{frame}
Multi-Staged Programming: What is it?
\end{frame}
\end{centering}

\begin{frame}
Normal Model: Running a program on some input produces some output
\end{frame}

\begin{frame}
Staged Model: Running a program on some input produces some output \textit{which might be a program}
\end{frame}

\begin{frame}
Staged Model: Running a program on some input produces some output \textit{which might be a program}\\
which then can be run on some other input, to produce some more output
\end{frame}

\begin{frame}
Staged Model: Running a program on some input produces some output \textit{which might be a program}\\
which then can be run on some other input, to produce some more output \textit{which might be a program}\\
\end{frame}

\begin{frame}
Staged Model: Running a program on some input produces some output \textit{which might be a program}\\
which then can be run on some other input, to produce some more output \textit{which might be a program}\\
which then can be run on some other input, to produce some more output
\end{frame}

\begin{frame}
Staged Model: Running a program on some input produces some output \textit{which might be a program}\\
which then can be run on some other input, to produce some more output \textit{which might be a program}\\
which then can be run on some other input, to produce some more output \textit{which might be a program}\\
\end{frame}

\begin{frame}
Staged Model: Running a program on some input produces some output \textit{which might be a program}\\
which then can be run on some other input, to produce some more output \textit{which might be a program}\\
which then can be run on some other input, to produce some more output \textit{which might be a program}\\
etc...
\end{frame}

%Intro to staging language
\begin{frame}{New language constructs}
\begin{itemize}
  \item $\langle ... \rangle$ means delay evaluation of enclosed code until the next stage. E.g. $\langle 5 * 2 \rangle$ is a program that when run produces the product of 5 and 2.
  \item \textasciitilde means evaluate this piece right away, and can only occur within $\langle \rangle$. E.g. $\langle 5 * $\textasciitilde $(1 + 1)\rangle$ evaluates to $\langle 5 * 2\rangle$ immediately.
  \item $!$ means take a program and run it. E.g. $!\langle 5 * 2 \rangle$ evaluates to 10.
\end{itemize}

Other than this, the language used is the traditional $\lambda$ calculus. This extended version is refered to as $\lambda ^U$
\end{frame}

\begin{frame}{Obligatory power function example}
\small
\texttt{let rec power n x = if n = 1 then x else x * power (n-1) x}\\
\texttt{let rec genpow n x = if n = 1 then x}\\
\texttt{\ \ \ \ \ \ \ \ else $\langle$ \textasciitilde x * \textasciitilde (genpow (n-1) x) $\rangle$}\\
\texttt{let stpow = ! $\langle$ fun z $\rightarrow$ \textasciitilde (genpow n $\langle$ z $\rangle$ ) $\rangle$}
\end{frame}

\begin{frame}{MSP Design Patterns}

\begin{itemize}
  \item Continuation Passing Style: needed for ``name'' (binding)
    generation for loops and conditionals
  \item Monads: code generation can be viewed as an effect; Monad can
    aid in generating binding statements for loops and conditionals
  \item Functors: help make code more staged code more modular
\end{itemize}

See ``Multi-Stage Programming with Functors and Monads: Eliminating
Abstraction Overhead from Generic Code''

\end{frame}

\begin{frame}{CPS Example}
\small

\texttt{let one = .$\langle$ 1 $\rangle$. and plus x y =
  .$\langle$.\textasciitilde x  + .\textasciitilde y $\rangle$.}

\vspace{3mm}

Want to generate code that generates code the $(y + 1) +
(x + (y + 1))$, but with no redundant $(y + 1)$

\vspace{3mm}

\texttt{let letgen exp k = .$\langle$ let t = .\textasciitilde exp in
  .\textasciitilde (k  .$\langle$ t $\rangle$.) $\rangle$. }

\vspace{3mm}

\texttt{let param\_code plus one }

\texttt{  let gen x y k = }

\texttt{      letgen (plus y one) (fun ce -> k (plus ce  (plus x ce)))}

\texttt{  and k0 x = x}

\texttt{  in  .$\langle$ fun x y ->  .\textasciitilde (gen .$\langle$ x $\rangle$ .$\langle$ y $\rangle$. k0 $\rangle$.  }

\end{frame}



\begin{frame}{Staging Erasure}
Define the erasure of staging from an expression to be noted as $||e||$\\
\begin{itemize}
  \item $||x|| \defeq x$
  \item $||c|| \defeq c$
  \item $||\lambda x.e|| \defeq \lambda x.||e||$
  \item $||$\textasciitilde$e|| \defeq ||e||$
  \item $||e_1 e_2|| \defeq ||e_1|| ||e_2||$
  \item $||\langle e \rangle|| \defeq ||e||$
  \item $||!e|| \defeq ||e||$
\end{itemize}
\end{frame}

\begin{frame}{Calling Convention Review}
\begin{itemize}
  \item Call By Name: arguments are not evaluated before a function call, are syntactically inserted into the function body, and are instead evaluated at \textit{each place} within the function where they are needed.
  \item Call By Value: arguments are evaluated once before being passed to a function call, and the resulting value is substituted everwhere.
  \item Call By Need: arguments are not evaluated before a function call, but if needed more than once in a body their result is stored in a temporary and reused. This is what Haskell uses.
\end{itemize}
\end{frame}


\begin{frame}{CBV expressions are distinguished by context:}

Is $(\lambda\_.0) x$ the same at $0$?:

\begin{equation*}
\langle \lambda x.^\sim [(\lambda\_.\langle 1 \rangle) 0] \rangle
\Downarrow^0 \langle \lambda x.1 \rangle
\end{equation*}


But the following gets ``stuck'', because $x$ is bound to a variable
on level 1 (that hasn't been computed yet):

\begin{equation*}
\langle \lambda x.^\sim [(\lambda\_.\langle 1 \rangle)
 ((\lambda\_.0) 
 x)] \rangle \Uparrow^0
\end{equation*}

\end{frame}

\begin{frame}

A modified substitution ($\beta$ reduction rule) for CBV:

\begin{equation*}
\lambda x.C[(\lambda y.e^0) x] = \lambda x.C[[x/y]e^0]
\end{equation*}

assuming both contexts are not-staged:

\begin{equation*}
C[(\lambda y.e^0) x], C[[x/y]e^0]  \in E^0
\end{equation*}

\end{frame}

%%new beta reduction rule here maybe? somewhere else maybe? needs to go somewhere

\begin{frame}{Erasure Theorem, Call By Name Edition}
$\lambda _n ^U \vdash e \rightarrow ^* t$ implies $\lambda _n ^U \vdash ||e|| \rightarrow ^* ||t||$.\\
\vspace{1cm} 

That is, if we use the call by name, then terms that evaulate to some
value will always evaluate to that same value with the staging
annotations erased on both the term and the value. If a term halts,
its erased counterpart will also halt.
\end{frame}

\begin{frame}{Erasure Theorem, Call By Value Edition}
$\lambda _v ^U \vdash e \rightarrow ^* t$ implies $\lambda _n ^U \vdash ||e|| \rightarrow ^* ||t||$.\\
\vspace{1cm}

That is, if we use the call by value, then terms that evaulate to some
value will always evaluate to that same value with the staging
annotations erased on both the term and the value \textit{under call
  by name}. If a term halts, its erased counterpart will also halt
\textit{under call by name}.
\end{frame}

%%Commuting diagrams necessary here?

%%call by value example for stpow?

\begin{frame}{Applicative Bisimulation}
Extremely intuitive definition: Run two programs on the same input. If
they both halt and produce the same output (which could be other
programs that you again need to bisimulate), they are extensionally
equivalent.
\end{frame}



\end{document}
